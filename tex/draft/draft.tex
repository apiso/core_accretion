%\documentclass[12pt, preprint,numberedappendix]{emulateapj}
%\documentclass[12pt, preprint]{aastex}
\documentclass[apj]{emulateapj}

\newcommand\submitms{n}		% set to y to follow AAS ``ms'' names, etc.
\newcommand\bibinc{n}		% set to y if bib pasted in .tex, set to n to use bibtex


%\usepackage{pdfsync}
\usepackage{subeqnarray}
\usepackage{natbib}
\usepackage{color}


\bibliographystyle{apj}

\newcommand{\ie}{i.e.\ }
\newcommand{\eg}{e.g.\ }
\newcommand{\p}{\partial}
\newcommand{\brak}[1]{\langle #1\rangle}


\newcommand{\gcc}{\;\mathrm{g\; cm^{-3}}}
\newcommand{\gsc}{\;\mathrm{g\; cm^{-2}}}
\newcommand{\cm}{\; {\rm cm}}
\newcommand{\mm}{\; {\rm mm}}
%\newcommand{\ps}{\; {\rm s^{-1}}}
\newcommand{\km}{\; {\rm km}}
\newcommand{\au}{\; \varpi_{\rm AU}}
\newcommand{\AU}{\; {\rm AU}}
\def\K{\; {\rm K}}

\newcommand{\vcs}[1]{\mbox{\boldmath{$\scriptstyle{#1}$}}}
\newcommand{\vc}[1]{\mbox{\boldmath{$#1$}}}
\newcommand{\nab}{\vc{\nabla}}
\DeclareMathSymbol{\varOmega}{\mathord}{letters}{"0A}
\DeclareMathSymbol{\varSigma}{\mathord}{letters}{"06}
\DeclareMathSymbol{\varPsi}{\mathord}{letters}{"09}

\newcommand{\Eq}[1]{Equation\,(\ref{#1})}
\newcommand{\Eqs}[2]{Equations (\ref{#1}) and~(\ref{#2})}
\newcommand{\Eqss}[2]{Equations (\ref{#1})--(\ref{#2})}
\newcommand{\App}[1]{Appendix~\ref{#1}}
\newcommand{\Sec}[1]{Sect.~\ref{#1}}
\newcommand{\Chap}[1]{Chapter~\ref{#1}}
\newcommand{\Fig}[1]{Fig.~\ref{#1}}
\newcommand{\Figs}[2]{Figs.~\ref{#1} and \ref{#2}}
\newcommand{\Figss}[2]{Figs.~\ref{#1}--\ref{#2}} 
\newcommand{\Tab}[1]{Table \ref{#1}}

\definecolor{gray}{gray}{0.5}
\newcommand{\emgr}[1]{\emph{ \color{gray} #1}}


%\newenvironment{packed_item}{
%\begin{itemize}
%  \setlength{\itemsep}{1pt}
%  \setlength{\parskip}{0pt}
%  \setlength{\parsep}{0pt}
%}{\end{itemize}}

\begin{document}

\slugcomment{Draft Modified \today}


\shorttitle{Critical Core Mass at Wide Separations}
\shortauthors{Youdin \& Piso \emph{TBD}}

\title{On the Minimum Core Mass for Giant Planet Formation}
%\title{Lower Limits on the Core Mass for Giant Planets}
%\title{The Critical Core Mass of Wide-Separation Giant Planets:  Lower Limits from Kelvin-Helmholtz Cooling}
\author{Andrew N.\ Youdin}
\affil{JILA, University of Colorado}

\author{Ana-Maria Piso}
\affil{Harvard Smithsonian Center for Astrophysics}

%\begin{abstract}
%\end{abstract}

\section{Introduction}
\emgr{Background topics to mention include: core accretion history, giant planets at wide separations and theoretical challenges posed (GI probably too big, core accretion maybe to slow).}

\emgr{Describe goals and approach of this paper.}

This paper is organized as follows: \emgr{(section summaries)}.


\section{Atmosphere Models}\label{sec:model}

\subsection{Basic Assumptions}
\emgr{ Define and describe basic length scales; Describe disk conditions: MMSN etc. Describe basic assumptions: 1D atmosphere, smooth matching onto the disk, atmosphere composition etc.  \\
The idea is to start gentle with generalities and then get specific with the 2-layer quasistatic model after the basic eqns. have been given.}

%Describe assumptions specific to our model: quasistatic, $L_{KH} \gg L_{acc}$, two-layer, constant $L$. Explain how our model is intermediary between purely static models with $L=L_{acc}$ and full time-dependent evolutionary models. 

\subsection{Structure Equations}
\emgr{Give in radius coordinates. EOS (polytrope, various assumption for $\nabla_{\mathrm{ad}}$ and $\mu$) and opacity laws (noting that we only consider cool radiative zones with dust opacities). Give boundary conditions}

\subsection{Standard Methods of Solution} 
\emgr{shooting vs. Henyey methods to satisfy above BC. Static models powered by accretion and full evolutionary calculations. Note that there's nothing really to explain about the ``shooting method" .  Fine to state the BC's and that we use the shooting method.  That's a difference between a paper and research notes. We probably want the radius version of structure ODEs.}



\subsection{Virial Equilibrium and Global Cooling}
\emgr{Start with basic idea of quasistatic contraction and $L = -\dot{E}$, then write out cooling equation and explain terms.   Explain where it comes from (local cooling + virial equilibrium).   Defer some derivation to Appendix \ref{sec:virial}.  In principle this is still general stuff, i.e. the global cooling eqn applies to all hydrostatic models, not just our quasi-static one.}

Before describing our quasistatic method we first describe the general ways that planetary Kelvin-Helmholz contraction differs from the stellar case.

\subsection{Quasi-static Two-layer Model}
\emgr{Two layer model and simplified treatment of radiative zone. How to obtain an evolutionary series from cooling model between subsequent static atmospheres. }

\section{Analytic Cooling Model}
\emgr{derive model (perhaps deferring some results to an appendix), give and explain expected scalings.  Could go after numerical results, but if first can use to explain some numerical results.}



\section{Quasistatic Kelvin-Helmholz Contraction}
\emgr{Show instantaneous profiles. Also L-t, L-M evolution.}

\subsection{The End of Quasistatic Evolution}
\emgr{Describe that time / luminosity becomes negative when $M_{atm} \sim M_c$.}

\section{Critical Core Mass}
\emgr{These are the take home results! Define critical core mass (= minimum between mass doubling and entropy minimum). Explain the assumptions, show that growth is slowest then (M-t plot?)}

\subsection{Role of Disk Temperature and Pressure}
\emgr{Show how critical core mass changes as you vary one parameter while keeping the other constant.}

\subsection{Core Mass vs. Cooling Time at Fixed Radius}

\subsection{Core Mass vs. Disk Radius}

\subsection{Opacity Dependence}
\emgr{Not sure what would go in this section, since we only use one opacity model.}

%\subsection{EOS Tables}
%\emgr{compare polytrope vs. real EOS, this could go first or last.}




\section{Neglected effects}
\emgr{EOS, time evolution, planetesimal accretion, inefficient convection (mixing length models).  More detailed opacity laws (non-powerlaw, ice sublimation, extension to short period planets). }
\subsection{Planetesimal Accretion}
\emgr{Compare and contrast our results with Rafikov's, show accretion rates plot; discuss why our fixed core mass assumptions are valid in the regime we are considering.}


\section{Conclusions}



%\begin{figure}[tb!] %  figure placement: here, top, bottom, or page
%\if\submitms y
% 	\includegraphics[width=6in]{f.eps}
% \else
%	\hspace{-1cm}
%   	\includegraphics[width=3.9in]{../figs/.pdf} 
%      \includegraphics[width=3.9in]{../figs/.eps} 
%\fi
%   	\caption{}
%   	\label{fig:}
%\end{figure}


\acknowledgements
\emgr{Acknowledge Ruth, Phil Armitage, Scott Kenyon, Jonathan Fortney \& Didier Saumon.}  Portions of this project were supported by the {\it NASA} {\it Astrophysics Theory Program} and  {\it Origins of Solar Systems Program}  through grant NNX10AF35G. \emgr{JILA, Ruth's grants?}\\

\appendix
%\section{Extending the EOS tables}\label{sec:EOSext}
\section{Derivation of Cooling Model}\label{sec:virial}
\section{Analytic Cooling Model Details}\label{sec:analytic}

\if\bibinc n
\bibliography{refs}
\fi

\if\bibinc y
\begin{thebibliography}
\end{thebibliography}
\fi

\end{document}