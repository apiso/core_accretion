% !TEX TS-program = pdflatexmk

\documentclass[12pt, preprint,numberedappendix]{emulateapj}
%\documentclass[12pt, preprint]{aastex}

\newcommand\submitms{n}		% set to y to follow AAS ``ms'' names, etc.
\newcommand\bibinc{n}		% set to y if bib pasted in .tex, set to n to use bibtex


\usepackage{pdfsync}
\usepackage{subeqnarray}
\usepackage{natbib}


\bibliographystyle{apj}

\newcommand{\ie}{i.e.\ }
\newcommand{\eg}{e.g.\ }
\newcommand{\p}{\partial}
\newcommand{\xv}{\vc{x}}
\newcommand{\kv}{\vc{k}}
\newcommand{\brak}[1]{\langle #1\rangle}


\newcommand{\gcc}{\;\mathrm{g\; cm^{-3}}}
\newcommand{\gsc}{\;\mathrm{g\; cm^{-2}}}
\newcommand{\cm}{\; {\rm cm}}
\newcommand{\mm}{\; {\rm mm}}
%\newcommand{\ps}{\; {\rm s^{-1}}}
\newcommand{\km}{\; {\rm km}}
%\newcommand{\au}{\; \varpi_{\rm AU}}

\newcommand{\AU}{\; {\rm AU}}
\newcommand{\yr}{\; {\rm yr}}
\def\K{\; {\rm K}}

\newcommand{\vcs}[1]{\mbox{\boldmath{$\scriptstyle{#1}$}}}
\newcommand{\vc}[1]{\mbox{\boldmath{$#1$}}}
\newcommand{\nab}{\vc{\nabla}}
\DeclareMathSymbol{\varOmega}{\mathord}{letters}{"0A}
\DeclareMathSymbol{\varSigma}{\mathord}{letters}{"06}
\DeclareMathSymbol{\varPsi}{\mathord}{letters}{"09}

\newcommand{\Eq}[1]{Equation\,(\ref{#1})}
\newcommand{\Eqs}[2]{Equations (\ref{#1}) and~(\ref{#2})}
\newcommand{\Eqss}[2]{Equations (\ref{#1})--(\ref{#2})}
\newcommand{\Eqsss}[3]{Equations (\ref{#1}), (\ref{#2}) and~(\ref{#3})}
\newcommand{\App}[1]{Appendix~\ref{#1}}
\newcommand{\Sec}[1]{Sect.~\ref{#1}}
\newcommand{\Chap}[1]{Chapter~\ref{#1}}
\newcommand{\Fig}[1]{Fig.~\ref{#1}}
\newcommand{\Figs}[2]{Figs.~\ref{#1} and \ref{#2}}
\newcommand{\Figss}[2]{Figs.~\ref{#1}--\ref{#2}} 
\newcommand{\Tab}[1]{Table \ref{#1}}

\newenvironment{packed_item}{
\begin{itemize}
  \setlength{\itemsep}{1pt}
  \setlength{\parskip}{0pt}
  \setlength{\parsep}{0pt}
}{\end{itemize}}

\newcommand{\delad}{\nabla_{\rm ad}}
\newcommand{\delrad}{\nabla_{\rm rad}}
\newcommand{\Rg}{\mathcal{R}}
\newcommand{\RB}{R_{\rm B}}
\newcommand{\co}{_{\rm c}}
\newcommand{\di}{_{\rm o}}
\newcommand{\cb}{_{\rm CB}}
\newcommand{\surf}{_M}
\newcommand{\mc}{m_{\rm c \oplus}}
\newcommand{\mcn}[1] { m_{ \rm c #1 \oplus} }
\newcommand{\MC}{M_{\rm crit}}
\newcommand{\au}{a_\oplus}
\newcommand{\aun}[1]{ a_{#1\oplus} }

\begin{document}

\slugcomment{Draft Modified \today}


\shorttitle{Core Accretion}
\shortauthors{Youdin }

\title{Core Accretion of Self-Gravitating Atmospheres}
\author{Andrew N.\ Youdin}
\affil{Harvard Smithsonian Center for Astrophysics}

%\begin{abstract}
%\end{abstract}
\section{Introduction}
The core accretion model \citep{PerCam74, MizNak78} proposes that giant planets form by the accretion of gas onto a solid protoplanetary core.    Early versions of the theory found a ``critical core mass", $\MC$, above which static solutions do not exist and unstable atmospheric collapse would occur \citep{Har78, Miz80}.  These works used hydrostatic models that include heating by planetesimal accretion, but neglect the heat generated by the gravitational, or Kelvin-Helmholtz (KH), contraction of the atmosphere.  

In evolutionary models that include KH contraction \citep{BodPol86}, the atmosphere does not undergo a hydrodynamic collapse.  Instead the quasi-static contraction of the envelope accelerates continuously.  Contraction becomes particularly rapid after the the  ``crossover" mass, when core and atmospheric masses are equal \citep{pollack96}.  The concepts of critical and crossover masses are similar, and even agree numerically when the planetesimal accretion luminosity exceeds the KH luminosity.  However if the planetesimal accretion rate is small and KH contraction dominates, the critical core mass will be smaller than the more relevant  crossover mass \citep{IkoNak00}.

This paper will consider 1D atmospheres  where KH contraction dominates over planetesimal accretion.  Such solutions give realistic lower limits on the core mass required to accrete a massive gaseous atmosphere.  Indeed even models that include planetesimal accretion obtain the fastest formation times by stopping planetesimal accretion after the core reaches a certain mass \citep{HubBod05}.

The early phases of atmospheric growth are usually treated with 1D models where the outer boundary of the atmosphere matches onto the conditions in the disk midplane.   As the atmosphere continues to grow, 1D solutions are no longer adequate.  The atmosphere both notices that the disk is not spherically symmetric and at large enough masses will start to open a gap in the disk.  In this case the accretion of gas is limited by the ability of the disk to supply gas to the growing protoplanet \citep{DAnLub08}.  This later stage evolution is crucial for determining the final mass of the planet, but  not considered in our calculations of the initial collapse.
 



%Analytic models to explain the critical core mass have their shortcomings. \cite{Ste82} assumed a fully radiative atmosphere, which is inconsistent with realistic luminosities and opacities.  \cite{Wuc93} assumed a fully convective atmosphere which is is inconsistent with the ability of atmospheres to cool.  \citet[hereafter R06]{Raf06} considerably advanced the analytic understanding of core accretion models.  (Explain what's great.)  However R06 neglected the luminosity due to gas accretion.  

%Most analyses of the critical core mass considers planetesimal accretion as the dominant luminosity source.  One notable exception is \cite{PapNel05}.  There are several reasons to consider the atmospheres on protoplanets that are not accreting solids.  First the planetesimal accretion rate may be a negligible source of luminosity.  (In this sense our results complement those of R06).  Second the growth of an atmosphere around a preexisting core represents lower limit to the timescale to form a giant planet.  Since atmospheric mass grows sharply with core mass, there may be little loss in assuming that atmospheric accretion begins once core formation stops.  Finally while it is desirable to self-consistently solve for core growth and atmospheric accretion, it is impossible to account for all possible core growth processes, which are quite complicated (refs).  
%For instance planet migration or scattering could bring a core that formed rapidly in one region of the disk to a region of the disk where cores could not form rapidly.


Section \ref{sec:structure} describes the structure of planet atmospheres embedded in a gas disk.   We then derive the critical core mass to trigger runaway collapse of the atmosphere in \S\ref{sec:critmass}.

\section{Disk Model}
As a fiducial disk model, we adopt the version of the minimum mass, passively irradiated model of \citet{cy10}, which gives the surface density and midplane temperature as
\begin{subeqnarray}
\label{eq_sigmag}
\varSigma  &=& 70 \,F_\varSigma \aun{10}^{-3/2} ~{\rm g~cm}^{-2} \\
T &=& 45  \,F_T\, \aun{10}^{-3/7} ~{\rm K}
\end{subeqnarray} 
where $\aun{10} \equiv a/{10~\rm AU}$ scales the disk radius and the normalization factors $F_\varSigma$ and $F_T$ adjust the disk mass and temperature relative to the canonical values.   The resulting disk midplane pressure 
\begin{equation}
P\di = 6.9 \times 10^{-9} F_\varSigma \sqrt{F_T m_\ast} \, \aun{10}^{-45/14}~{\rm bar}
\end{equation} 
for a mean molecular weight of 2.35.  The stellar mass in solar units is $m_\ast \equiv M_\ast/M_\odot$, which only includes the gravitational effects, though stellar mass also influences temperature and probably correlates with average disk mass.  Away from the midplane the pressure drops with height $z$ as a Gaussian with scale-height
\begin{equation}
H = {\sqrt{\Rg T\di} \over \varOmega} = 0.42 \sqrt{F_T \over m_\ast}  \, \aun{10}^{9/7} \AU\, .
\end{equation} 


\section{Atmospheric Structure}\label{sec:structure}
In this section we review the structure of atmospheres surrounding a solid protoplanetary core embedded in a gas disk.  The core has mass $M\co$ and radius $R\co = [3 M\co/(4 \pi \rho\co)]^{1/3}$ with internal density $\rho\co \approx 2~{\rm g~cm^{-3}}$.  For protoplanetary disk parameters we use the minimum mass, passively irradiated disk model presented in \citet[hereafter  CY10]{cy10}.  The disk has a midplane gas density $\rho\di$ and temperature $T\di$.

The core can attract a dense atmosphere inside its Bondi radius
\begin{equation} 
R_B = {G M\co \over \mathcal{R} T\di} \approx 0.17 {\mcn{10}  \, \aun{10}^{3/7} \over F_T} ~\AU
\end{equation} 
where $\mcn{10} \equiv M\co/(10~M_\oplus)$ and $\mathcal{R}$ is the specific gas constant.

Low mass planets satisfy a hierarchy of length scales,  $R\co \ll R_B < R_{\rm H} < H$, where %$h\di \approx 3.3 \times 10^6 \au^{9/7}~{\rm km}$ is the gas scale-height and
\begin{equation} 
R_{\rm H} = \left(M\co \over 3 M_\ast \right)^{1/3}a \approx 0.22 {\mcn{10}^{1/3} \, \aun{10} \over m_\ast^{1/3}}~\AU
\end{equation} 
is the Hill radius, inside which the planet's gravity competes with Keplerian shear.

As protoplanets grow in mass the length scales $R_{\rm H} \sim R_{\rm B} \sim H$ are roughly equal at a characteristic ``transitional" mass
\begin{equation}
M_{\rm tr} > {(\Rg T\di)^{3/2} \over G \varOmega} \approx 25 \, {F_T^{3/2} \over \sqrt{m_\ast} } \, \aun{10}^{6/7}~ M_\oplus \, .
\end{equation} 
Technically the length scales re-order in three steps of increasing mass.  First for $M > M_{\rm tr}/\sqrt{3}$ the ordering becomes $R_{\rm H} < R_{\rm B} < H$ as tidal truncation becomes more important than thermal constriants.  While the shear flow past the atmosphere is clearly not spherically symmetric, it is still common take the atmosphere as spherically symmetric with an outer boundary at the Hill radius.  When the mass increases further to $M > M_{\rm tr}$ then  $R_{\rm H} < H < R_{\rm B}$, which does not change the fact that tidal truncation is the stronger effect at the outer boundary. 

Finally when $M > 3 M_{\rm tr}$ the high mass scaling $H < R_{\rm H} < R_{\rm B}$ is attained.  The assumption of spherical symmetry fails for two reasons.  First the planet now notices the vertical density stratification of the disk.  Second, the planet's torques on the disk can begin to open gaps, creating further density inhomogeneity.  Since runaway accretion typically begins for masses well below $\sim 75 M_\oplus$, the assumption of spherical symmetry is a good, if not perfect, approximation for understanding whether cores can attrach massive atmospheres.  Moreover the approximation becomes even better in the disk's outer regions.

Another relevant for atmospheric structure is the mass of disk gas that fills the Bondi radius, at the ambient density.  The ratio of this mass to the core mass is 
\begin{equation}
\theta\co \equiv { 4 \pi \rho\di R_B^3 \over 3 M\co} \approx 5 \times 10^{-3} {\mcn{10}^2 \over \aun{10}^{3/2}}{F_\varSigma \sqrt{m_\ast} \over F_T^{7/2}}\, .
\end{equation} 
The smaller this parameter the more gravitational compression is required for an atmosphere to become self-gravitating.

There are two limiting cases of atmospheric structure, isothermal and adiabatic.  Atmospheres that are isothermal at $T\di$ have cooled completely and give the most massive atmosphere.   Convectively unstable atmospheres transport heat very efficiently and are nearly adiabatic.  Adiabatic atmospheres are the opposite extreme of the isothermal case, giving the steepest increase of temperature with atmospheric dept.  The least dense atmospheres are fully adiabatic with the entropy of the disk. 

The structure of an embedded protoplanet consists of an inner adiabatic region that matches onto an outer radiative zone, that in some cases, can be approximated as nearly isothermal.  This basic structure was demonstrated by R06 for an atmosphere supported by planetesimal accretion luminosity.  PN05 showed that atmospheres powered by gas accretion alone have a similar structure.  Due to their intense irradiation, ``hot Jupiters" similarly have an isothermal layer that matches onto a deeper adiabatic interior \citep{ab06, ym10}.  

\subsection{Isothermal Atmosphere}
We now consider the structure of a low mass, i.e.\ non-self gravitating, isothermal atmosphere.  We assume the atmosphere matches onto a constant background density, $\rho\di$, at a distance $r_{\rm fit} = n_{\rm fit} \RB$.  The resulting density profile is
\begin{equation} \label{eq:rhoiso}
\rho = \rho\di \exp \left({R_B \over r} - {1 \over n_B} \right) \approx   \rho\di \exp \left(R_B \over r  \right)
\end{equation} 
where the approximate inequality holds deep inside the atmosphere ($r \ll \RB$) for any $n_{\rm fit} \gtrsim 1$.  However the choice of boundary condition does have an order unity effect on the density near the Bondi radius.   

The mass of the atmosphere typically determined by integrating the density profile from the core to the Bondi radius.  Planets can attract massive atmospheres if $\theta\co \equiv R_B/R\co \gg 1$.  In this case
\begin{equation} \label{eq:MatmISO}
M_{\rm iso} \approx 4 \pi \rho\di {R\co^4 \over R_B} e^{R_B/R\co} = 4 \pi \rho\di {R\co^3 \over \theta\co} e^{\theta\co}\, .
\end{equation} 
This result is the leading order term in a series expansion.  Furthermore, because the atmospheric scale-height at $R\co$ is $H_\rho = |dr /d\ln\rho| = R\co^2/R_B$, the result is intuitively the correct order of magnitude.

From \Eq{eq:MatmISO} the ratio of isothermal atmosphere to core mass is 
\begin{equation} 
{M_{\rm iso} \over M\co} = 3 {\rho\di \over \rho\co}{R\co \over R_B} e^{R_B/R\co} %= {\theta_o \over \theta\co^2} e^{\theta\co}
\end{equation} 
While  $\rho\di \ll \rho\co$, the exponential dependence allows the atmosphere to become massive.  
%A value of $\theta\co = R_B / R\co$ corresponds to a planet mass of
%\begin{equation} 
%M\co = \sqrt{3 \over 4 \pi \rho\co} \left(\theta\co \mathcal{R} T \over G \right)^{3/2} 
%\end{equation} 
As a concrete example, consider a disk with $T = 60$ K and $\rho\di = 3 \times 10^{-11}$ (in cgs) and a core density $\rho\co = 2$.  The atmosphere self-gravitating for $\theta\co = 27.1$ or $M\co > 0.046 M_\oplus$.  The fact that very low mass cores can have very massive isothermal atmospheres is well known \citep{Sas89}.  However such solutions are unrealistic because there is inadequate time to radiate away all the atmosphere's thermal energy.

We not consider the mass exterior to the Bondi radius.  For a meaningful evaluation we only include the mass coming from the overdensity relative to the background density.  The resulting external mass for an isothermal atmosphere is
\begin{subeqnarray}
M_{\rm ext} &=& 4 \pi \int_{\RB}^{r_{\rm fit}} (\rho - \rho\di) r^2 dr \\
&=& M\co \theta\co \int_1^{n_{\rm fit}} 3 \left[ \exp \left({1 \over x} - {1 \over n_{\rm fit}}\right) - 1 \right] x^2 dx  \nonumber \\
&\equiv& M\co \theta\co I(n_{\rm fit})
\end{subeqnarray} 
where the dimensionless integral $I(n_{\rm fit})$ obeys the limits $I(1) = 0$ and $I \rightarrow n_{\rm fit}^2/2$ as $n_{\rm fit} \rightarrow \infty$.  Since this external mass does not converge the choice of an outer boundary does matter in principle.  In practice however the assumption that   $r_{\rm fit} = R_{\rm H}$ limits $n_{\rm fit}$ to modest values
\begin{equation}
n_{\rm fit} = {R_{\rm H} \over \RB} \approx 1.3 {\aun{10}^{4/7} \over \mcn{10}^{2/3}}{F_T \over  m_\ast^{1/3}}
\end{equation} 
since for instance $I(2) = 1.1$ these modest $n_{\rm fit}$ values will only produce a small external mass.

However the effect is still worth including for the effect on the interior mass.  
\subsection{Temperature Contrast at Convective Boundary}
We now show that the temperature contrast between the convective boundary, $T\cb$, and the ambient disk $T\di$, is modest.  We express the radiative lapse rate
\begin{equation}\label{eq:delrad}
\delrad \equiv {d \ln T \over d \ln P} = {3 \kappa P \over 64 \pi  G M \sigma T^4} L = \nabla\di {(P/P\di)^{1+\alpha} \over (T/T\di)^{4-\beta}}
\end{equation}
where $M$ is the sum of the core mass and the atmosphere mass below the pressure level $P$.  If the mass in the radiative zone is small, then we can hold $M = M\co + M_{\rm conv}$ fixed at the sum of core and convective zone masses.  With this assumption and powerlaw opacity, we get a constant value for $\nabla\di$.  The temperature profile then integrates to
\begin{equation}\label{eq:radTP}
\left(T \over T\di\right)^{4-\beta} - 1 = {\nabla\di \over \nabla_\infty} \left[\left({P \over P\di}\right)^{1-\alpha} - 1 \right] \, ,
\end{equation} 
where $\nabla_\infty = (1+\alpha)/(4-\beta)$ is $\delrad$ for $T ,P \rightarrow \infty$.\footnote{Our definitions of $\nabla\di$ and $\nabla_\infty$ are precisely opposite to R06, but consistent with other works and the general convention of labeling a quantity $f$ in the disk as $f\di$.}
We  apply \Eqs{eq:delrad}{eq:radTP} at the convective boundary $\delrad = \delad$ under the assumption that the pressure there is $P\cb \gg P\di$. The resulting temperature contrast at the convective boundary is
\begin{equation}\label{eq:Tcb}
\chi \equiv {T\cb\over T\di} \simeq \left(1 - {\nabla_{\rm ad} \over \nabla_\infty}\right)^{-\frac{1}{4-\beta}} \, .
\end{equation} 
For  the low temperatures in protoplanetary disks, opacity is dominated by dust with $\alpha = 0$ and $\beta \approx 2$ or $1$, giving $\chi \approx 1.5$ or 1.9, respectively, assuming  $\delad = 2/7$.

\subsection{Location of Convective Boundary}
The pressure at the convective boundary  follows from \Eqs{eq:delrad}{eq:Tcb} as
\begin{equation}
{P\cb\over P\di} \simeq \left({\delad/\nabla\di \over 1 - \delad/\nabla_\infty}\right)^{1 \over 1 + \alpha}
\end{equation} 
This pressure contrast can be quite large due to the smallness of $\nabla\di$ in low luminosity atmospheres.
 
 It is also useful to obtain a relation between $T$ and $P$ that eliminates $\nabla\di$ in favor of $P\cb$:
 \begin{equation}\label{eq:TP}
{T \over T\di} = \left\{1 + {1 \over {\nabla_\infty \over \delad} - 1} \left[ \left({P \over P\cb}\right)^{1-\alpha} -  \left({P\di \over P\cb}\right)^{1-\alpha}  \right] \right\}^{1 \over 4-\beta}\, .
\end{equation} 
 
We can determine the radius of the convective boundary $r\cb$ from the hydrostatic balance equation as 
\begin{equation}\label{eq:RCBint}
{R_B \over r\cb} = \int_{P\di}^{P\cb} {T \over T\di} {dP \over P}\, .
\end{equation} 
An isothermal atmosphere gives a simple logarithmic dependence on $P\cb$.  However using \Eq{eq:TP} in the integral gives
\begin{equation}\label{eq:Rcb}
{R_B \over r\cb} = \ln \left(P\cb \over P\di \right) - \ln \theta \, ,
\end{equation} 
with an extra correction term, $\theta < 1$.  In the $P\cb \gg P\di$ limit the correction term is a constant that depends on $\alpha$, $\beta$ and $\delad$.  The form of $\theta$ was chosen to allow us to express
\begin{equation}\label{eq:PcbRcb}
P\cb = \theta P\di e^{R_B/r\cb} \, .
\end{equation}   
A simple analytic expression for $\theta$ is not possible.  


\begin{deluxetable}{cccccc}  % <--- column justification (center/left/right)
\gdef \numcols {6}
\tablecolumns{\numcols}
\tablecaption{Parameters Describing Structure of Radiative Zone.}
\tablehead{   \multicolumn{\numcols}{c} {$\gamma = 7/5$ ($\delad = 2/7$), $\alpha = 0$} }  
\startdata
 $\beta$   		 &1/2  	& 3/4 &1   		& 3/2  		& 2   \\
 $\nabla_\infty$ & 2/7 \tablenotemark{a}  	&  4/13	& 1/3 	& 2/5 	 	& 1/2 \\
 $\chi$ 		 & \nodata &  2.25245 &1.91293 	& 1.65054 	& 1.52753 \\
 $\theta $  		 &\nodata   & 0.145032	&0.285824   &0.456333   & 0.556069   \\
\enddata
\tablenotetext{a}{Since $\delad = \nabla_\infty$ there is no convective transition at depth for this case.}
\end{deluxetable}



The T-P profile can also be combined with the equation of hydrostatic balance to relate the radius and pressure of the convective boundary.  If the convective boundary is well inside the Bondi radius, i.e.\ $r\cb\ll R_B$ and $P\cb\gg P\di$ then
\begin{equation}\label{eq:PcbRcb}
P\cb = \theta P\di e^{R_B/r\cb} \, .
\end{equation}    
The order unity constant $\theta < 1$ accounts for the fact that the radiative layer is not perfectly isothermal, as $\theta$ would be exactly one in that case.  The value of theta depends on the equation of state and the pressure and temperature dependence of the opacity, i.e. the $\alpha (=0)$ and $\beta$ values.  We find $\theta(\beta = 2) \approx 0.556$ and $\theta(\beta = 1) = 0.286$.  Closed form expressions for these integrals are too cumbersome to be useful.


\section{Virial Theorem}
The virial theorem for self-gravitating atmospheres can be applied to protoplanetary atmospheres.    The virial theorem is derived as usual by integrating the equation of hydrostatic balance to give
\begin{eqnarray}\label{eq:virial}
E_G= -3 \int_{M\co}^{M} {P \over \rho}dm + 4 \pi \left( R^3 P\surf - r\co^3 P\co\right) 
\end{eqnarray} 
where the gravitational energy
\begin{equation}\label{eq:EG}
E_G = -\int_{M\co}^{M} {G m \over r} dm \, .
\end{equation} 
At the outer surface $m = M$, $r = R$ and other values are given a subscripted $M$.  This outer surface can be evaluated anywhere (e.g.\ Bondi radius, Hill radius or RCB) that hydrostatic balance holds.  The boundary terms include the effects of finite surface pressure and finite radius of the solid core.

 The integral on the RHS \Eq{eq:virial} is related to the internal energy.  For an ideal gas, $P/\rho = \Rg T = (\gamma -1) u$, where $u$ is the internal energy per mass, and $\gamma = C_P/C_V$.  For a polytrope with constant $\gamma$ we can express the virial theorem in terms of internal energy $E_i = \int u dm$ as
\begin{equation}
E_G = - \zeta E_i + 4 \pi \left(R^3 P\surf - r\co^3 P\co\right) 
\end{equation} 
where $\zeta \equiv 3 (\gamma -1)$.  In this case the total energy $W = E_G + E_i$ becomes
\begin{subeqnarray}
W &=& (1-\zeta)E_i + 4 \pi \left( R^3 P\surf  - r\co^3 P\co \right) \\
 &=& {\zeta - 1 \over \zeta} E_G + {4 \pi \over \zeta} \left( R^3 P\surf  - r\co^3 P\co \right)\, .
\end{subeqnarray} 
The surface pressure acts to unbind the atmosphere by increasing $W$.  While a realistic EOS is neither ideal nor polytropic, \Eq{eq:virial} is general.


%\section{Adiabatic Atmosphere}
%An adiabatic atmosphere with the entropy of the disk gas is only slightly more massive than the disk gas itself (inside the Bondi radius).

\section{Cooling Models}
An isolated sphere satisfies a simple global energy equation
\begin{equation}\label{eq:simplecool}
L = \Gamma - \dot{E}
\end{equation} 
where the luminosity is balanced by the rate of heat generation, $\Gamma$, and the rate at which total (gravitational plus thermal, at least) energy is lost.

For a protoplanetary atmosphere embedded in a gas disk, the cooling equation is more complicated: 
\begin{equation}\label{eq:EB}
L = L\co + \Gamma- \dot{E} + e_{\rm acc} \dot{M}  -  P\surf {\p V\surf \over \p t} \, .
\end{equation}
We explain these extra terms and then derive the result.   \Eq{eq:EB} assumes a well (but arbitrarily) defined surface at radius $R$ and mass $M$, which can evolve in time.  Examples include the Bondi radius or the radiative-convective boundary.  The luminosity $L$ represents the luminosity from that surface.  The luminosity from the solid core is given by $L\co$, and includes planetesimal accretion energy  and radioactive decay.  The core is assumed to have a fixed mass, $M\co$, and radius, $R\co$, to simplify the atmospheric calculation.  The $\Gamma$ and $\dot{E}$ integrals are identical to \Eq{eq:simplecool}, and include the atmosphere from the core to the top boundary.   Sources of direct heating that  contribute to $\Gamma$ include  atmospheric drag on sedimenting planetesimals and the dissipation of any atmospheric turbulence.

Mass accretion at the rate $\dot{M}$ brings specific energy $e_{\rm acc} = u - GM/R$, the sum of internal and gravitational energies.  The energy of accreted matter, $e_{\rm acc}$, is  zero at a modified Bondi radius, $R_{\rm out} = G M/u \approx  (\Rg/C_V) R_B$.  For $R < R_{\rm out}$ ($R > R_{\rm out}$, respectively) the accreted mass has negative (positive) energy and is (un)bound.
 The final term in \Eq{eq:EB} gives the pressure work done on a surface mass element, so the partial time derivative of  volume holds the mass fixed.  Our quasistatic models we neglect bulk kinetic energy.   

\subsection{Derivation of Global Energy Equation}
We now derive the global energy equation (\ref{eq:EB}).  We follow the simpler example in \S4.3 of \citet{KipWei94}, adding the effects of finite core radius, surface pressure and mass accretion.  We assume that hydrostatic balance holds.  Integrating the local energy equation from core to surface gives:
\begin{subeqnarray}
L - L\co &=& \int_{M\co}^M {\p L \over \p m} dm \\
&=& \int_{M\co}^M \left(\epsilon - T {\p S \over \p t} \right)dm \\
&=& \Gamma  - \int_{M\co}^M{\p u \over \p t} dm +  \int_{M\co}^M {P \over \rho^2} {\p \rho \over \p t} dm\slabel{eq:DLc}\, .
\end{subeqnarray} 
with $\Gamma = \int \epsilon dm$ the integral of the direct heating rate.

In what follows, we must carefully distinguish between partial time derivatives, $\p / \p t$, (performed at fixed mass) and total time derivatives, denoted with overdots (which include the effect of mass accreted through the outer boundary). % or $d/dt$.  
For instance the evolution of surface radius, $R$, evolves as  
\begin{equation}\label{eq:Rdot}
 \dot{R} = {\p R \over \p t} + {\dot{M} \over 4 \pi R^2 \rho\surf}
\end{equation} 
where $\p R/\p t$ gives the Lagrangian contraction of surface mass elements, and $\dot{M}$ denotes mass accretion rate through the upper boundary.  The subscript $M$  denotes quantities at the upper boundary of total mass $M$ (though it is omitted from $M$ and $R$).  Similarly the volume, $V = (4 \pi/3)r^3$ and pressure at the surface evolve as
\begin{subeqnarray}\label{eq:dot}
\dot{V}_M &=&  {\p V_{\rm M} \over \p t} + {\dot{M} \over \rho_{\rm M}}  \\
 \dot{P}_M &=& {\p P_{\rm M} \over \p t} + {\p P_M \over \p m}\dot{M} \\
 &=&  {\p P_{\rm M} \over \p t} - {G M  \over 4 \pi R^4} \dot{M}\, .
\end{subeqnarray} 
For the purpose of this derivation we will hold the core mass fixed $\dot{M}\co = 0$ which further gives $\dot{P}\co = \p P\co / \p t$.

To derive the global energy equation we must move the (partial) time derivatives in \Eq{eq:DLc} outside their integrals.  The internal energy integral follows simply from  Leibniz's rule as
\begin{equation}\label{eq:udot}
\int_{M\co}^{M(t)}{\p u \over \p t} dm = \dot{E}_i  -  \dot{M}u\surf\, .
\end{equation} 

To evaluate the work integral, we derive a pair of expressions for the rate of change of gravitational energy.
The time derivative of \Eq{eq:virial}  gives
\begin{eqnarray}\label{eq:EGdot}
\dot{E}_G &=& 3  \int_{M\co}^M {P \over \rho^2} {\p \rho \over \p t} dm -3 \int_{M\co}^M {\p P\over \p t}{dm \over \rho} \\
&& -  3{P\surf \over \rho\surf} \dot{M}+ 3 \dot{P}\surf V\surf -3 \dot{P}\co V\co  + 3  P\surf {\dot{ V}\surf} \, . \nonumber
%&&+ 4 \pi \left. \left(r^3 {\p P \over \p t}\right)\right|_{M\co}^M  + 3  P\di {\p V \over \p t}  - {G M \over R} \dot{M} \nonumber
\end{eqnarray} 
%where the volumes, $V\surf = 4 \pi R^3/3$ and $V\co = 4 \pi R\co^3/3$. 
The first integral in \Eq{eq:EGdot} is the one we want, but the next one must be eliminated.  The time derivative of \Eq{eq:EG} (times four) gives
\begin{subeqnarray}
 4 \dot{E}_G &=&  -4 {G M \dot{M} \over R} + 4 \int_{M\co}^M {G m \over r^2}{\p r \over \p t} dm\\ 
&=&   -4 {G M \dot{M} \over R} + 4 \pi \int_{M\co}^M r^3{\p \over \p m}{\p P \over \p t} dm \slabel{eq:4EGb} \\
&=&  -4 {G M \dot{M} \over R} -3  \int_{M\co}^M {\p P\over \p t}{dm \over \rho} \slabel{eq:4EGc} \\
&&+ 3 V\surf {\p P\surf \over \p t} -3 V\co {\p P\co \over \p t} \nonumber 
\end{subeqnarray} 
where \Eqs{eq:4EGb}{eq:4EGc} use hydrostatic balance  and integration by parts.

%To eliminate the time derivates of pressure, we take the time derivative of the hydrostatic balance equation for $\p^2 P / \p m\p t$ and integrate over $4\pi r^3 dm$ (as in the virial equation derivation) to get
%\begin{equation}\label{eq:dHBdt}
%3 \dot{P}\surf V\surf -3 \dot{P}\co V\co -3 \int_{M\co}^M {\p P\over \p t}{dm \over \rho}  = 4 \dot{E}_G + 4{G M \over R} \dot{M}  \, .
%\end{equation} 
%Combining \Eqs{eq:EGdot}{eq:dHBdt} gives 
%\begin{eqnarray}\label{eq:rhodot}
%\int_{M\co}^M {P \over \rho^2} {\p \rho \over \p t} dm  &=& - \dot{E}_G - {4 \over 3}{G M\over R} \dot{M} + {P\surf \over \rho\surf} \dot{M} -  P\surf \dot{V}\surf  \, , \nonumber \\
%&=&- \dot{E}_G - {4 \over 3}{G M\over R} \dot{M}  -  P\surf {\p V\surf \over \p t}  \, ,
%\end{eqnarray} 
%where the final step uses \Eq{eq:Rdot}.

Subtracting \Eqs{eq:udot}{eq:4EGc} and rearranging for the desired integral gives
\begin{eqnarray}\label{eq:PdVint}
\int_{M\co}^M {P \over \rho^2} {\p \rho \over \p t} dm  &=&  - \dot{E}_G - {G M \dot{M} \over R} - P\surf {\p \dot{V}\surf \over \p t} \,  
\end{eqnarray} 
with \Eq{eq:dot} used to combine terms.  Combining \Eqsss{eq:DLc}{eq:udot}{eq:PdVint}, we reproduce \Eq{eq:EB} with the accreted energy $e_{\rm acc} \equiv u\surf - GM/R$.  



\section{Simplified cooling model}
We consider an analytic cooling model that assumes an ideal gas polytrope for the convective region and neglects self-gravity.  

\subsection{Luminosity,  Energy \& Mass}
The luminosity that emerges at the radiative convective boundary is
\begin{equation} \label{eq:Lcb}
L\cb = {64 \pi G M\cb \sigma T\cb^4 \over 3 \kappa P\cb } \nabla_{\rm ad} \approx L\di {P\di \over P\cb} % \left({P\di \over P\cb}\right)%^{1+\alpha}
\end{equation} 
for a dust opacity with $\kappa \propto T^\beta$ and
\begin{equation} 
L\di \equiv {64 \pi G M\cb \sigma T\di^4 \over 3 \kappa(T\di) P\di} \nabla_{\rm ad}\chi^{4-\beta}\, .
\end{equation} 
normalized to disk conditions, with $T = \chi T\di$ at the RCB.

We adopt an opacity law
\begin{equation}
\kappa = 2 F_\kappa \left(T \over 100~{\rm K}\right)^\beta ~{\rm cm^2/g}
\end{equation} 
independent of pressure as appropriate for dust opacities.  For $\beta = 2$ and $F_\kappa = 1$, this gives the \citet{bl94} opacity for icy grains.  By varying $F_\kappa$, dust depletion or enhancement can be considered.  Grain properties affect both $F_\kappa$ and $\beta$ which generally satisfies  $1/2 \lesssim \beta \lesssim 2$ (aside from discontinuities across sublimation regions), see \citet{SemHen03}.

To make further analytic progress, we ignore the self-gravity in the convective zone, holding the mass fixed at $M\co$.  The density profile of an adiabatic atmosphere follows from hydrostatic balance as 
\begin{equation}\label{eq:rhoconv} 
\rho = \rho\cb \left[ 1 + {R_B' \over r} - {R_B' \over r\cb}  \right]^{1/(\gamma -1)}\, .
\end{equation} 
where we define an effective Bondi radius,
\begin{equation}
R_B' \equiv {G M\co \over C_P T\cb} = {\delad \over \chi} R_B
\end{equation} 
to simplify expressions.

  Deep in the atmosphere, where $r \ll r\cb \lesssim R_b'$ the density profile is $\rho \propto r^{-1/(\gamma -1)}$.  Since energy scale as $\rho r^2 \propto r^{(2\gamma -3)/(\gamma - 1)}$ only polytropes with $\gamma < 3/2$ (i.e. $\gamma = 7/5$, but not $\gamma = 5/3$) have the bulk of energy at the bottom of the atmospheres.  We will thus focus on the $\gamma = 7/5$ case, even though dissociation occurs deep in real protoplanetary atmospheres.

The total (thermal and gravitational) energy in an adiabatic atmosphere could be evaluated from the virial theorem.  More simply, we use the result for temperature profiles in deep (but non-self-gravitating) convective regions:
\begin{equation}
T \approx {G M\co \over C_P r} = T\cb {R_B' \over r}\, .
\end{equation} 
The internal energy per unit mass of an ideal gas is thus $u = C_V T = (1 - \delad) G M\co/r$ and the specific energy  deep in the atmosphere is
\begin{equation}
e = e_g + u = -\delad {GM\co \over r} \, .
\end{equation} 
Thus the total energy for $\gamma < 3/2$ is thus
\begin{eqnarray} 
E &=& - 4 \pi \nabla_{\rm ad} G M\co \int_{R\co}^{r\cb} \rho r dr \\
&\approx& - 4 \pi P\cb {R_B'}^{1 \over \nabla_{\rm ad}} \left(\gamma-1 \over 3 - 2 \gamma\right)  R\co^{2\gamma-3\over \gamma-1}  \\
&\approx& - 8 \pi P\cb {R_B'^{7/2} \over \sqrt{R\co}}
%&\approx& -0.31 P\cb {(R_B / \chi)^{7/2} \over \sqrt{R\co}}
%&\approx& -0.31 P\cb {R_B'^{7/2} \over \sqrt{R\co}}
\end{eqnarray} 
where the final expression takes $\gamma = 7/5$. % and then revert to the standard Bondi radius $R_B = G M\co/(\Rg T\di)$ in terms of the disk temperature.

The mass of the adiabatic atmosphere is given by
\begin{eqnarray} 
M_{\rm atm} &=& 4 \pi \int_{R\co}^{r\cb} \rho r^2 dr \\
&=& {5 \pi^2 \over 4} \rho\cb {R_B'}^{5/2} \sqrt{r\cb} % \approx 0.54 \rho\cb {R_B'}^{5/2} \sqrt{r\cb} \\
%&=& 0.54{R_B'^{3} \sqrt{\chi} \over  \sqrt{ \ln[P\cb/(\theta P\di)]}} \, .
\end{eqnarray}
in the limit $R\co \ll r\cb \ll R_B'$. The mass in the isothermal region is $M_{\rm iso} \sim 4 \pi \rho\cb r\cb^4/R_B \ll M_{\rm atm}$, and can be neglected.


We can eliminate $r\cb$ with \Eq{eq:PcbRcb}.  The ratio of atmosphere to core mass is then  
\begin{equation} \label{eq:crit}
{M_{\rm atm} \over M\co} \approx {P\cb /P_M \over  \sqrt{\ln[ P\cb/(\theta P\di)]}}
\end{equation} 
where we introduce a characteristic pressure
\begin{equation} 
P_M \equiv {4 \delad^{3/2} \over 5 \pi^2 \sqrt{\chi} } {G M\co^2 \over {R_B'}^4}\, .
%P_M \approx {1.85 \over \sqrt{\chi}} {G M\co^2 \over {R_B'}^4}\, .
\end{equation} 

For the atmosphere to become self-gravitating, with $M_{\rm atm} = M\co$, we thus require
\begin{equation} \label{eq:Pcbc}
P\cb = \xi P_M
\end{equation} 
where  the logarithmic factor
\begin{equation}\label{eq:xi}
\xi \equiv \sqrt{\ln[ P\cb/(\theta P\di)]} = \sqrt{\ln[ \xi P_M / (\theta P\di)]}
\end{equation} 
is found by numerically solving the above transcendental equation. The physical solution has $\xi >1$, but typically order unity. Numerical solutions exit for $\xi \ll 1$, but are unphysical as they imply $P\cb < P\di$.
%  The physical origin of this logarithm is the exponential increase of pressure with depth in the radiative zone.

\subsection{Cooling Time}
Neglecting surface terms, the total time to cool the atmosphere from an initially fully adiabatic state is
\begin{eqnarray} 
t_{\rm  cool} &=& -\int {d E \over L} = - \int_{P\di}^{P\cb } {d E/d P\cb \over L} dP\cb \\
&\approx& 4 \pi {P\cb^{2} \over P\di} {R_B'^{7/2} \over L\di \sqrt{R\co}} \label{eq:tcool}\, ,
%&\approx& {0.16 }{P\cb^{2} \over P\di} {R_B'^{7/2} \over L\di \sqrt{R\co}}
\end{eqnarray} 
where we neglect self-gravity and take $M\cb = M\co$ in $L\di$.

The cooling timescale  for an atmosphere to become self-gravitating is found from \Eqs{eq:tcool}{eq:Pcbc} as
\begin{eqnarray} 
t_{\rm cool} &\approx & 2 \times 10^8 {F_T^{5/2}  F_\kappa \left(\xi \over 3.4\right)^2  \over \left(\mc \over 10 \right)^{5/3} \left(\au \over 10\right)^{15 \over 14}} \yr
\end{eqnarray} 
for $\beta = 2$.  
Clearly this timescale is too long, and this is likely due to missing physics of self-gravity and EOS.  Nevertheless the trends are informative.   A reduction in opacity naturally gives faster cooling, as long as the optically thick assumption holds.  Lower disk temperatures also give faster cooling.  Even though higher temperatures give higher luminosities, they also give higher dust opacities and lower the Bondi radius and gas density.  In balance higher temperatures suppress cooling in this model.    For different $\beta$ values, the temperature dependence is $t_{\rm cool} \propto F_T^{1/2 + \beta}$.  The cooling timescale only weakly depends on disk mass or pressure, via the  logarithmic factor $\xi$.  This weak dependence is a consequence of cooling coming from the convective boundary.  Note that the scaling laws do not reflect the changes to $\xi$ which remains order unity and is here given the appropriate value for the nominal parameters.



\subsection{Critical Core Mass}\label{sec:critmass}
%We use the results of the previous section to derive the minimum core mass that can form a giant planet by gas accretion.  Since our structure calculations do not include the self-gravity of the atmosphere, we simply assume that runaway growth occurs once the atmosphere's mass (as calculated in the absence of self-gravity) reaches the core mass, $M_{\rm atm} \ge M_{\rm c}$.  This approximation was used by R06 and is consistent with detailed evolution models (refs).


We can define a critical more mass as that which gives a  self-gravitating atmospheres within a typical gas disk lifetime:
\begin{equation}
t_{\rm cool} = 3 \times 10^6 \tau_{\rm cool} ~{\rm yr} \, ,
\end{equation} 
with $\tau_{\rm cool}$ a scaling factor.  The resulting critical core mass is
\begin{equation}\label{eq:Mcrit}
M_{\rm crit} \approx 100 {F_T^{3/2} F_\kappa^{3/5}   \left(\xi \over 2.6 \right)^{6/5} \over \left(\au \over 10\right)^{9 \over 14}} \; M_\oplus
\end{equation} 
as with the cooling time, the critical mass is too large due to missing physics.  The scaling with disk properties is similar, in fact all quantities are simply raised to the $3/5$ power.  The $\xi$ value is changed to match the value for the nominal solution.  The lower value reflects the fact that an atmosphere around such a massive core does not need to contract as much to be self-gravitating.  

The numerical values for a different opacity law do not change the numerical answers above very much.  That is mostly because we have chosen a disk location where $T_o$ is not far from 100 K, where our opacity law is normalized.  More generally a higher $\beta$ value is more favorable for core accretion at large distances due to the sharper drop in opacity.

%We use \Eq{eq:tcool} to find $P\cb$, the pressure depth of the convective boundary after a cooling time.   Relative to the disk pressure we find
%\begin{equation}
%{P\cb\over P\di} =  2.0 \times 10^4 \left(a \over 5~{\rm AU}\right)^{2.0} \mc^{-7/6} {\sqrt{\tau_{\rm cool}} \over F  \sqrt{\kappa_\odot}}
%\end{equation} 


%This seems totall wrong:
%\begin{equation} 
%{M\co \over M_\oplus} \approx 14 \left(a \over 5~{\rm AU}\right)^{0.19} \left[{\kappa_s (\xi/2)^3 f_T^4 \over F \sqrt{M_\ast}}\right]^{3/20}
%\end{equation} 

\subsection{The Opacity Effect}
A  lower opacity  could lower the core mass.  Reducing the opacity by a factor of one hundred cuts the core mass by more than a factor of 10, specifically to 9 $M_\oplus$ for the parameters in \Eq{eq:Mcrit}.  The reduction is not  as strong as the nominal scaling would imply, $0.01^{3/5} \approx 0.06$, because $\xi$ increases.

Even with significantly lower opacities, radiative diffusion remains a good approximation at the convective boundary.  We estimate the optical depth as (for $\beta = 2$):
\begin{equation}
\tau\cb \sim {\kappa P\cb \over g} \sim 7 \times 10^4 {F_T^4 F_\kappa \over \left(\mc \over 10 \right) \left(\au \over 10\right)^{12 \over 7}} 
\end{equation} 
where $P\cb \sim P_M$ for a self-gravitating atmosphere and $g \sim G M\co/R_B^2$, with both approximation good to within the the order unity factor $\xi$.  Clearly $\tau\cb \gg 1$ even for $F_\kappa \lesssim 0.01$ out to very wide separations.

A hotter disk would increase core masses.  Instead of our passive disk model, adopting the standard Hayashi temperature profile would increase core masses by $\sim 50\%$.  A hotter accretion phase would further increase core masses, but such phases are presumably short lived.

\subsection{Surface Terms}
We now check the relevance of the neglected surface terms in \Eq{eq:EB}.  We already showed that accretion energy can be exactly eliminated by choosing an outer boundary near the Bondi radius.   We now show that accretion energy is also a small correction at the RCB.   A rough comparison, (ignoring terms of order $\xi$) of  accretion luminosity vs.\ $\dot{E}$ gives
\begin{equation}
{G M \dot{M} \over R \dot{E}} = {G M  \over R}{dM \over dE} \sim {G M\co^2 \over R_B E}{P\cb \over P_M} \sim \sqrt{R\co \over R_B} \ll 1
\end{equation} \, ,
where we assume $P\cb \sim P_M$  for a massive atmosphere.  Accretion energy at the protoplanetary surface is thus very weak for marginally self-gravitating atmospheres, and even weaker for lower mass atmospheres.  A similar scaling analysis shows that the work term, $P\surf \p V\surf/\p t$ is similarly weak.  Nevertheless numerical calculations should include, or check, the importance of these surface terms in more realistic self-gravitating atmospheres.


\section{Hydrogen Dissociation}
The dissociation of molecular Hydrogen deep in the atmospheres of accreting protoplanets plays a significant role in the energetics of core accretion.  In the high density regions $r  \ll r\cb$ of a convective atmosphere, the thermal plus gravitational energy scales as
\begin{equation}
dE = -4 \pi \nabla_{\rm ad}^{1/\nabla_{\rm ad}} \rho\cb R_B'^{1/(\gamma-1)} r^{\frac{2\gamma - 3}{\gamma - 1}} {dr \over r}
\end{equation} 
If $\gamma < 3/2$ then the main contribution to the energy is at the bottom of the atmosphere, i.e.\ the core.  This is the case for a diatomic ideal gas ($\gamma = 1.4$) or a solar mixture of hydrogen and helium ($\gamma \approx 1.43$).  However a monatomic gas has $\gamma = 5/3 > 3/2$.  In this case, the atmosphere's energy is no longer concentrated near the bottom, but will be concentrated near the top of the convective zone.

A likely structure is an atmosphere that is dissociated near the base, but becomes molecular near the top of the convective zone.  In this case the atmosphere's energy budget would be concentrated near the atomic to molecular transition.

The energy required to dissociate a hydrogen molecule, $I = 4.467$ eV can be significant to the overall energy budget.
Since
\begin{equation}
{I \over \nabla_{\rm ad} G M\co (2 m_{\rm H})/r} \approx 3 \left(M\co \over 10 M_\oplus \right)^{-2/3} {r \over R_{\rm c}}
\end{equation} 
we see that this energy is always relevant.

We can use the Saha equation to determine where dissociation is significant,
\begin{equation}
{n_{\rm H}^2 \over n_{\rm H_2}} = \left(\pi m_{\rm H} k T \over h \right)^{3/2} e^{-I/(kT)}
\end{equation} 
We introduce a reaction coordinate $\delta$ so that $n_{\rm H} = 2 \delta n_o$ and $n_{\rm H_2} = (1-\delta) n_o$ with $n_o = \rho X/(2 m_{\rm H})$ the number density when all hydrogen is molecular.  We express equilibrium as
\begin{equation}
{\delta^2 \over 1-\delta} = f_\mu {P_{\rm diss}(T) \over P}
\end{equation} 
with the characteristic pressure below which dissociation occurs is
\begin{equation}
P_{\rm diss} = {\left(kT\right)^{5/2} \over 4} \left( \pi m_{\rm H} \over h^2\right)^{3/2}  e^{-I/(kT)}
\end{equation} 
and the order unity factor
\begin{equation}
f_\mu = 2\delta + (1-\delta) + Y/2 + Z/\mu_Z
\end{equation} 
accounts for variations in the mean molecular weight with dissociation.  (Take $\mu_Z = 31/2$, but not too significant.)

Thus dissociation occurs where $P \lesssim P_{\rm diss}(T)$.  At disk temperatures (say 150 K) the dissociation pressure is negligibly small ($\sim 10^{-141}$ dyne cm$^{-2}$) and no dissociation occurs.  However at core temperatures the dissociation pressure is quite large especially for massive cores.  Dissociation is guaranteed.

%\section{What is the Core Instability}
%It is often argued that the critical core mass arises from the inability of a massive atmosphere to maintain hydrostatic equilibrium.    The analytic version of this argument in S82 makes several assumptions.  In particular self-gravity is neglected in some places, but included in others.   Presumably the assumptions are reasonable, because numerical  structure models of Mizuno and Papaloizou \& Terquem give similar results.
% The physical nature of the core instability remains unclear (to me at least).
%
%One possibly illuminating test problem would be the integration of the S82 problem (raditive diffusion at constant opacity) with a consistent inclusion of self-gravity.  The integration solves the structure equations for $dT/dr$, $dP/dr$ (or $d \rho /dr$) and $dM/dr$.  With self-gravity it's hard to avoid the numerical nature of the problem with the boundary conditions for $M$ applying at the core, but for $T$ and $P$ in the disk.
%
%One question is how self-gravity changes the temperature profile (and this is relevant for adiabatic profiles as well).  Without self gravity, the optically thick result is $T = G M\co/(4 R_g R\co)$.  S82 assumes the core mass is replaced by the total mass.  This may be reasonable in an average sense?

\section{Structure Models}
%We propse to numerically solve for the evolution of the atmosphere of a protoplanet that cools in the absense of planetesimal accretion.  Thus all heat loss comes form the atmosphere.  Since much of the atmosphere is convective most models require a mixing length model to calculate the luminosity.  Motivated by analytic models, the goal is to avoid the need for a mixing length model by simply calculating the radiative luminsotiy $L$ at a consistently determined RCB.  The goal then is to construct solutions along a sequence of entropy that gradually decreases from the disk entropy.  From a self-consistently determined luminosity, the cooling timescales would then be given consistently.  \footnote{I thought about fixing $L$ letting $S$ adjust, but this seems problematic if (for instance) $L$ is double-valued.} 
%
%A Henyey method would probably work best.  Practical concerns include a good initial guess and ``root finding" to determine $L$ at a given $S$.  Radial windows that cause multiple convective zones are also a concern. 

The atmospheric structure equations  in pressure units are
\begin{subeqnarray}\label{eq:struct}
{\p m \over \p P} &=& - {4 \pi r^4 \over G m} \\
{\p r \over \p P} &=& - { r^2 \over G m \rho} \\
{\p T \over \p P} &=&  \nabla {T \over P} \\
{\p L  \over \p P} &=& {\p m \over \p P} \epsilon\, .\slabel{eq:dL}
\end{subeqnarray} 
When the atmospheric luminosity is carried by radiative diffusion, the lapse rate $\nabla$ is
\begin{equation}
\delrad ={ 3 \kappa P \over 64 \pi G m \sigma T^4}L\, .
\end{equation} 
We assume efficient convection when $\delrad > \delad$ so that $\nabla = \min(\delad, \delrad)$.
The local the local energy generation, $\epsilon$, includes heating from  gravitational contraction
\begin{equation}\label{eq:epsg}
\epsilon_g = -T {\p S \over \p t}\, .
\end{equation} 
Other sources of local heating, such as drag on infalling planetesimals could be specified but are not considered here.  To solve these equations we must supply opacity law for $\kappa$ and an equation of state (EOS) to give $\rho$ and $\delad$ as functions of $T$ and $P$ and the assumed composition.  Note that the luminosity and opacity do not affect the structure of convection zones, but do affect their boundaries.  If we can assume that the atmospheric luminosity is primarily generated in the convective zone, then the time dependence in the structure equations can be neglected.  

Boundary conditions for the problem exist both at the core and at the outer accretion radius.  At the core, $m = M\co$, $r = R\co$, $L = L\co$.  At the top boundary $T = T\di$ and $P = P\di$ set by conditions in the disk midplane.  In addition to these four boundary conditions, an initial solution is required for the time dependence that arises when  \Eq{eq:epsg} is included and allowed to affect the radiative zone structure.



\subsection{Shooting Method (1a)}\label{sec:shoot1a}
This is the first (and currently only) method we have numerically developed.  It is not described yet!  It is similar to method 1b below, but holds entropy values, $S_i$ fixed instead of mass (or luminosity).  In brief, we obtain solutions by varying the free thermodynamic variable at the core until the radius at the top of the convective boundary matches the value in \Eq{eq:Rcb}.

\subsection{Shooting Method (1b)}\label{sec:shoot1b}
We here describe an algorithm to obtain static solutions by integrating downward from the top of the convective zone.    As in \S\ref{sec:shoot1a} this method uses simplified solutions for the  radiative zone that neglect self gravity and luminosity generation.  With this method we can obtain an evolutionary sequence at fixed values of either mass or luminosity, with the other quantity optimized to get a converged solution.   Fixing the mass intervals has the advantage that total mass increases monotonically, while the luminosity is known to decrease to a minimum value and then increase (as with entropy in \S\ref{sec:shoot1a}). 

 The algorithm proceeds as follows:
\begin{enumerate}
\item Guess a value for either the luminosity, $L_{\rm guess}$, or mass, $M_{\rm guess}$, at the top of the convective zone (whichever is not being held fixed).  From this guess get estimates of $P\cb$ from \Eq{eq:Lcb}, $T\cb$ from \Eq{eq:Tcb} and $r\cb$ from \Eq{eq:Rcb}.
\item Integrate the structure equations from the convective boundary to the core, thereby obtaining a model value of the core mass. 
\item Repeat,  varying either $L_{\rm guess}$ or  $M_{\rm guess}$ in step 1 until the core mass in the atmospheric solution matches the assumed value.
\end{enumerate}

Solutions for a range of masses (and luminosities) can be connected in an evolutionary sequence using the global energy equation that we derive from the virial theorem.

For simplified radiative zone solutions, shooting from the top has the advantage that a monotonic sequence of mass can easily be contructed.  On the other hand, integrating from the bottom can be more numerically efficient because the tabulated values of thermodynamic variables along a single adiabat can be used for each solution.  When shooting from the top, the variation of the luminosity (or mass) requires integration along different adiabats, with greater interpolation costs.  (But computational costs aren't really a big limitation.?)

\subsection{Shooting Method (2)}\label{sec:shoot2}
%At this writing, we are using a shooting model (not described here) which uses the radiative convective boundary (RCB) as the fitting point.    The advantage of this approach is that a 1D (Newton Raphson) optimization of a thermodynamic variable at the core (such as temperature at fixed entropy) can be used to find a model.  A limitation of this approach is that the structure of the radiative zone must be computed analytically, making it difficult to include detailed physics, such as variable opacities and self-gravity (because the total mass of the atmosphere is unknown \emph{a priori}).  To overcome these obstacles we consider here a shooting method that extends from the core to the disk and the fitting is done at the outer boundary.  The price of this more detailed model is that a 2D optimization must be performed which includes finding the self-consistent luminosity.

%In traditional stellar models it is difficult to shoot to the outer boundary because of the numerical instability of very low pressures, i.e.\ small numerical errors in the high density region propagate like a whip (Hansen \& Kawaler; Kippenhahn \& Weigert).   These numerical instabilities should (hopefully) be less severe in the protoplanetary problem due to the finite core radius (and lack of central divergences) and the finite bounding pressure of the disk.

Here we describe a method that allows us to add the self-gravity of the radiative zone.  While a time dependent model could be developed (see \S\ref{sec:timedep}),  we first describe the method for constant radiative zone luminosity.  As in \S\ref{sec:shoot1b} we integrate inwards, this time starting from the disk. The outer boundary conditions are $T = T\di$,  $P = P\di$,  $M = M_i$ where $i$ is simply a label for each evolutionary stage, 
\begin{equation}
r = r_{\rm fit} = n_{\rm fit}{ G M_i \over \Rg T\di}\, ,
\end{equation} 
which places the outer radius at a finite number, $n_{\rm fit} > 1$, of Bondi radii.  Analytic solutions like \Eq{eq:rhoiso} show that the precise choice of fit radius has negligible effect on the atmospheric solution (R06).  

To find a solution we must vary $L = L_{\rm guess}$,  though as in \S\ref{sec:shoot1b} we could fix $L$ and vary  $M$.   An inward integration for a trial $L_{\rm guess}$  gives the value of the core radius $R_{\rm c, guess}$ implied by the trial solution.  We adjust $L_{\rm guess}$ until $R_{\rm c, guess}$ converges to the actual $R\co$.  

With atmospheric self-gravity, shooting outwards is more difficult because two parameters must be varied.  It is impossible to fix the total mass shooting from the bottom, but we can try to construct models for choices of $L_i$.  In this case,  both $P\co$ and $T\co$ must be varied to match $T = T\di$ and $r = r_{\rm fit}$ (using the total mass $M$ from the trial integration itself) at the top, which is defined by $P = P\di$.  Alternatively one could try to find solutions with fixed values of entropy $S = S_i$ for the interior adiabat as in \S\ref{sec:shoot1a}.  This gives only one thermodynamic variable to specify at the core, since the trial choice of $T\co$ gives $P\co$ at fixed entropy.  However trial values of $L\cb$ must also be chosen, and there remain two quantities to vary ($L\cb$ and the core thermodynamic variable) to match the two conditions at the upper boundary.  With only a single variable to optimize, downward integration is more efficient.  

\subsection{Time dependent shooting}\label{sec:timedep}
We now consider how to add time dependence to the method of \S\ref{sec:shoot2}.  \emph{For now this is just a sketchy description.}  First a static solution must be obtained for some small mass.  Then the next, slightly larger mass should be chosen.    A solution now requires variation of two quantities, the luminosity and the assumed time difference between the two states.  The time difference is used to calculate the heating in the convective zone using \Eq{eq:epsg}.  Convergence occurs when the atmosphere solution produces both the correct total mass and the timestep matched the global energy change between the two states. 

%\begin{eqnarray} 
%y(\xi\co, L)  &=&  T(P\di; \xi\co, L) - T\di\\
%z(\xi\co, L) &=& r(P\di; \xi\co, L) - n_{\rm fit}{ G m_{\rm B}(\xi\co, L) \over \Rg T\di}
%\end{eqnarray} 
%then the boundary conditions are met when $y = z = 0$.  The 2D Newton-Raphson method can be used to search for the optimal values of $\xi\co$ and $L$.  The mass enclosed by the Bondi radius, $m_{\rm B}$, is where the mass first exceeds $r \Rg T\di/G$.  In practice $m_{\rm B}$ will probably differ only slightly from $m(P\di)$ because of the low density of disk material, but this can be checked numerically.  


%\begin{figure}[tb!] %  figure placement: here, top, bottom, or page
%\if\submitms y
% 	\includegraphics[width=6in]{f.eps}
% \else
%	\hspace{-1cm}
%   	\includegraphics[width=3.9in]{../figs/.pdf} 
%      \includegraphics[width=3.9in]{../figs/.eps} 
%\fi
%   	\caption{}
%   	\label{fig:}
%\end{figure}

\acknowledgements
%Portions of this project were supported by the {\it NASA} {\it Astrophysics Theory Program} and  {\it Origins of Solar Systems Program}  through grant NNX10AF35G.\\
\appendix
\section{Self-Gravitating Isothermal Atmosphere}

For an isothermal atmosphere, hydrostatic equilibrium gives:
\begin{equation} \label{eq:HBiso}
r^2 {d \ln \rho \over d r} = -{G \over \mathcal{R} T} \left(M\co + 4 \pi \int_{R\co}^r \rho(r') r'^2 dr'\right) 
\end{equation} 
This integro-differential equation is subject to the boundary condition that the density is $\rho\di$ at large distances.  This boundary condition can be applied at $n_B R_B$, where the Bondi radius $R_B = GM\co/\mathcal{R} T$.  The number $n_B > 1$ does not affect the solution since the gravitational perturbation from the planet is weak outside the Bondi radius.

Radial differentiation of \Eq{eq:HBiso} gives:
\begin{equation} 
x^2 {d^2 S \over dx^2} + 2x {d S \over dx} = - \theta_o e^S x^2
\end{equation} 
where $S = \ln(\rho/\rho_o)$, $x = r/R\co$ and $\theta_o = 4 \pi G \rho\di R\co^2 /(\mathcal{R} T)$.  Since $\theta_o$ is the squared ratio of the core radius to the disk's Jeans length, $\theta_o \lll 1$. The outer boundary condition now reads $S(n_B \theta\co) = 0$ where $\theta\co = R_B/R\co > 1$.  An inner boundary condition
\begin{equation} 
{d S \over dx}(x = 1) = - \theta\co
\end{equation} 
must also be satisfied.


\if\bibinc n
\bibliography{refs}
\fi

\if\bibinc y
\begin{thebibliography}
\end{thebibliography}
\fi

\end{document}