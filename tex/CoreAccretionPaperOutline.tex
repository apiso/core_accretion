%\documentclass[apj]{emulateapj}
%\documentclass[manuscript]{aastex} 
\documentclass[aps,prd,preprint]{revtex4}


\usepackage{graphicx}                        
\usepackage{amsmath}
\usepackage{hyperref}
\usepackage{amsfonts}
\usepackage{amsmath}
\usepackage{amssymb}
\usepackage{amsthm}


%%%%%%%%%%%%%%%0


\begin{document}

\title{Core Accretion Outline}
\author{Ana-Maria Piso}

\begin{enumerate}
\item Introduction \& Motivation
\item Atmosphere model
\begin{enumerate}
\item Two-layer model - inner convective + outer radiative
\begin{enumerate}
\item describe analytical solution assuming ideal gas + no self-gravity?
\end{enumerate}
\item Basic equations
\item Boundary conditions + disk model
\item Real EOS (tables + table extension, most content of 'EOSnotes')
\item Virial theorem, energy budget, how to estimate cooling time
\end{enumerate}
\item Numerical method
\begin{enumerate}
\item Assumptions
\item Shooting method description
\end{enumerate}
\item Results \& Analysis
\begin{enumerate}
\item Plots: P-T, M-T, M-R etc.; Luminosity plots
\item Critical core mass
\begin{enumerate}
\item Mcrit vs. r plots, explain, comparison with polytropes
\item Mcrit vs. r for different disk assumptions?
\end{enumerate}
\end{enumerate}
\item Summary \& Conclusions
\begin{enumerate}
\item Caveats (neglecting self-gravity of radiative region, assumption of constant luminosity)
\item Future prospects: 2D extension
\end{enumerate}
\end{enumerate}




\end{document}